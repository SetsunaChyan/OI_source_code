\begin{problem}{雪山症候群}{雪山症候群.in}{雪山症候群.out}{1 seconds}

我不知道设局引诱我们来到这栋雪中怪屋的是何许人也,但是我绝对不会原谅害长门发高烧病倒的家伙。也不会让那种恶烂星人称心如意!无论如何,我们都会离开这里。

长门已经克尽她的职责。虽然途中我没看到也没听到,但是自从闯入这个异空间之后,她肯定一直在和看不到的“敌人”作战。她的表情显得比平常更加木然,想必就是那个原因造成的。虽然她战到鞠躬尽瘁,还是为我们开了个小小的风口。那么,接下来就轮到我们自己来打开这扇门了。

“我们要离开这里。”

门上赫然映着三行十进制阿拉伯数字,幸运的是长门告诉了我们解读门密码的规律。

第一行有 $n$ 个互不相同的正整数,记为 $a_1 ... a_n$ 。第二行和第三行分别有 $m$ 个正整数,记为 $b_1 ... b_m$ 和 $c_1 ... c_m$ 。

把 $b_i$ 和 $\{ a_1,...,a_n\}$ 中每个数的异或值算出来,其中第 $c_i$ 小的异或值就是密码的一部分。

然后依次把这 $m$ 个异或值连起来就是密码啦。

救救孩子吧。

\InputFile

第一行两个正整数 $n,m (1 \leq n,m \leq 10000)$,表示这三行数字的个数。

第二行 $n$ 个互不相同的正整数 $a_i (1 \leq a_i \leq 2^{60})$ 。

第三行 $m$ 个正整数 $b_i (1 \leq b_i \leq 2^{60})$ 。

第四行 $m$ 个正整数 $c_i (1 \leq c_i \leq n)$ 。 

\OutputFile

输出一行字符 $s$ ,表示门的密码。

\Example
\begin{example}
\exmp{
5 6
1 2 3 4 5
3 4 5 6 5 7
2 2 3 4 2 3
}{
114514
}%
\end{example}
\end{problem}
