\begin{problem}{射手座之日}{射手座之日.in}{射手座之日.out}{1 seconds}


黑暗的宇宙空间正在我眼前扩散开来。

那是有如戴上眼罩迷失在马头星云的黑暗空间,连一道星光都观测不到的乏味银河,说穿了就是偷工减料的布景。

这时,要是能加点演出就更好了的想法掠过我心头,不过凡事背后都有其成因,就连这个宇宙空间也是。像是预算、技术或时间等诸如此类的原因。

“什么都看不见嘛。”

我开始抱怨。屏幕的色彩从刚才就是清一色的黑,不禁让我怀疑显示器是否故障了。

当我正在思索自己究竟是在这个宇宙空间的哪里徘徊时,虚无的画面下半部突然出现了一个光点,而且开始埋头前进,我终于忍不住向上级陈情。

“喂,春日。要不要退后一点?你的旗舰太超前了。”

“作战参谋,请称呼我阁下。SOS团团长以军阶而言,少说也是上将级的大将之流。是我们当中最伟大的。”

在我回嘴谁是作战参谋谁又是阁下来着之前——

“凉宫阁下,长门情报参谋传来消息,说敌军舰队有可疑行动。请问该如何应对?”

狗腿军师古泉报告军情。春日的回答是:

“没关系,突击就对了!”

春日目前有着 $n$ 艘独立成队的战舰(编号从 $1$ 开始),春日可以把任意两艘战舰所在的队伍合并(即使它们已经被摧毁了),古泉也会报告第 $x$ 艘战舰被摧毁。作为一名泉水指挥官,春日想要知道某个时刻某个队伍里存活的战舰的数量。

注意,如果要合并的战舰已经在一个队伍里了,或者报告被摧毁的战舰在此之前已经被摧毁了,那么你可以无视他们的指令。

\InputFile

第一行两个正整数 $n,m (1 \leq n \leq 50000 , 1 \leq m \leq 50000)$ ,表示目前有 $n$ 艘独立成队的战舰,以及春日会发号施令 $m$ 次。

接下来 $m$ 行,每一行至少两个正整数 $op_i,x_i (op_i \in \{ 1,2,3 \} , 1 \leq x_i \leq n)$ 。

如果 $op_i==1$ ,接下来会有一个额外的正整数 $y_i (1 \leq y_i \leq n )$ ,表示春日想要合并 $x_i$ 和 $y_i$ 号战舰所在的队伍。

如果 $op_i==2$ ,表示 $x_i$ 号战舰被摧毁了。

如果 $op_i==3$ ,表示你需要回答春日 $x_i$ 号战舰所在的队伍里还有多少艘存活着。

\OutputFile

对于每一个询问,输出一行一个整数表示回答。

\Examples

\begin{example}
\exmp{
5 9
1 1 2
1 1 3
3 1
2 2
3 1
1 4 5
2 4
1 2 4
3 5
}{
3
2
3
}%
\end{example}
\end{problem}
