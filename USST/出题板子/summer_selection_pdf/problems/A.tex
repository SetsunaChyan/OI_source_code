\begin{problem}{Haruhi with Acm-hcpc}{standard input}{standard output}{1 second}

兵库県立西宫北高等学校马上要举办第一届 $ACM$-$HCPC$ 了,$Haruhi$ 已经迫不及待想要拿冠军了,现在“邀请”你编写一份滚榜程序。

注意你的滚榜程序必须满足以下要求,不然你会被 $Haruhi$ $\cdots$ 。

1.  组委会会提供输入文件,文件提供了封榜前的榜,和封榜后的提交信息。格式详见输入格式。

2.  关于排名:

\qquad i. 解出题目数多的队伍排名较前;

\qquad ii. 解出题目数相同,则总罚时较少的队伍排名较前;

\qquad iii. 解出题目数和总罚时都相同,则最后解出题目的时间更早的队伍排名较前。

3.  关于罚时:第 $i$ 题对罚时的贡献 $T_{i}=20\space *$ 第一次 $AC$ 前错误的次数 $+$ 第一次 $AC$ 的时间。总罚时 $=\sum T_{i}$ 。也就是说,\textbf{只有在第一次通过该题时才结算罚时}。

由于糟糕的评测平台,提交代码的结果只有 $WA(WrongAnswer)$ 和 $AC(Accepted) $ 两种。

$Haruhi$ 对谁过了哪些题不感兴趣,所以你只需要按排名输出队名、过题数以及罚时,输入保证没有并列。

\InputFile

第一行两个正整数 $n,m(1 \leq n \leq 1000,1 \leq m \leq 15)$ ,分别表示参赛队伍的数量和题目的数量。

接下来 $3*n$ 行。

其中的第 $3*k-2$ 行一个字符串 $s(1 \leq |s| \leq 100)$ 表示第 $k$ 支队伍的名字,只包含可见字符。 

第 $3*k-1$ 行有 $m$ 个字符串,表示第 $k$ 支队题目的提交信息,格式如下。

\begin{itemize}
\item $+p_{k,i}$ 表示 $WA$ 了 $p_{k,i}$ 次后 $AC$ 了第 $i$ 题。特别地,单独一个 $+$ 表示第一次提交就 $AC$ 。
\item $-p_{k,i}$ 表示 $WA$ 了 $p_{k,i}$ 次后仍未 $AC$ 第 $i$ 题。
\item  $0$ 表示未曾提交这题。
\item  其中 $1 \leq p_{k,i} \leq 100$。
\end{itemize}

第 $3*k$ 行有 $m$ 个整数 $q_{k,i}(-1 \leq q_{k,i} \lt 240)$ ,表示第 $k$ 支队第一次 $AC$ 了第 $i$ 题时的提交时间。数据保证 $q_{k,i}=-1$ 当且仅当该队没有通过此题。

再接下来一行一个整数 $h(0 \leq h \leq 1000)$ ,表示封榜后有 $h$ 次提交。

接下来 $h$ 行,每行四个正整数 $x,y,z,t(1 \leq x \leq n,1 \leq y \leq m,z \in\{0,1\},240 \leq t \lt 300)$ ,表示第 $x$ 支队伍在第 $t$ 分钟提交了 $y$ 题评测结果为 $z$ ,其中 $z=1$ 表示 $AC$ ,$z=0$ 表示 $WA$ 。

\OutputFile

按排名输出 $n$ 行,每一行包含一个字符串 $s$ 和两个正整数 $solved,penalty$ ,用一个空格分隔开,分别表示队伍的名字、过题数和罚时。

\Examples
\begin{example}
\exmp{
5 4
SOS\_Dan
-1 -2 0 +
-1 -1 -1 30
jxcyxctxc
-4 -5 -6 -7
-1 -1 -1 -1
I\_AK\_IOI
+ + + +
10 110 70 30
Sasaki
-1 -1 -1 -1
-1 -1 -1 -1
Pony.AI
+ + + +
50 60 70 40
10
1 1 1 250
1 2 1 255
1 3 0 256
1 3 1 257
1 3 1 260
2 1 1 244
2 2 1 299
3 1 1 240
4 1 0 241
4 2 0 242
}{
Pony.AI 4 220
I\_AK\_IOI 4 220
SOS\_Dan 4 872
jxcyxctxc 2 723
Sasaki 0 0
}%
\end{example}
\end{problem}
